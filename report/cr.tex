% !TeX root = cr.tex

\documentclass[12pt]{article}
\usepackage{svg}
\usepackage[utf8]{inputenc}
\usepackage[T1]{fontenc}
\usepackage[french]{babel}
\usepackage{fancyhdr}
\usepackage{lastpage}
\usepackage{graphicx}
\usepackage{eurosym}
\usepackage{amsmath}
\usepackage{multicol}
\usepackage{geometry}
\usepackage{nccrules}
\usepackage[table]{xcolor}
\usepackage{wrapfig}
\usepackage{pgfgantt}
\usepackage{makecell}
\usepackage{amssymb}
\usepackage{gensymb}
\usepackage{textcomp, gensymb} 
\usepackage{enumitem}
\usepackage{lmodern}
\usepackage{mathrsfs}
\usepackage{textcomp}
\usepackage{multicol}
\usepackage{listings}
\usepackage{media9}
\usepackage{graphicx}
\usepackage{breakurl}
\usepackage{parskip}
\usepackage{float}
\usepackage{listings}
\usepackage{color}
\usepackage{matlab-prettifier}
\PassOptionsToPackage{hyphens}{url}\usepackage{hyperref}
\usepackage{bookmark}

\usetikzlibrary{angles,calc,decorations.pathreplacing}

\title{\textbf{Projet image couleur 2 - Auto White Balance}}

\author{
    \textsc{Duteyrat Antoine},
    \textsc{Sève Léo}
}

\date{\today}

%----------------------------------------------------------------%
%----------------------------------------------------------------%
%----------------------------------------------------------------%

\definecolor{couleur}{RGB}{0,0,0}
\pagestyle {fancy}

\makeatletter
\let\titre\@title %Variable titre
\let\auteurs\@author %Variable auteurs
\let\date\@date %Variable date
\makeatother


%----------------------------------------------------------------%

%En-tête
\renewcommand{\headrulewidth}{1pt} %Taille du trait
\setlength{\headheight}{45pt}
\fancyhead[L]{\titre}
\fancyhead[R]{}

%Pied de page personnalisé :
\renewcommand{\footrulewidth}{0.5pt} %Taille du trait
\fancyfoot[C]{\thepage\ / \pageref{LastPage}} %PageActuelle / nbrePages au centre

%-----------------------------------------------------------------%
%-----------------------------------------------------------------%
%-----------------------------------------------------------------%

\begin{document}

%-----------------------%
%-----Page de garde-----%
\begin{titlepage}
    \begin{center}
        \vskip 1.5cm
        {\color {couleur}{\huge \bf \titre}}\\[5mm] % Affiche la variable titre
        \vskip 0.5cm
        \begin{figure}[h]
        \centering
        \includegraphics[width=7cm]{images/logo_tse.png}
        \end{figure}
        \vskip 1cm % Saut de ligne
        {\large \auteurs}\\ % Affiche la variable auteurs  
        \vskip 0.5cm % Saut de ligne
        \vfill
        \color{couleur}{\dashrule[1mm]{15cm}{0.5}} % Trait final
        \vskip 0.2cm
        \date % Affiche la variable date
      \end{center}
\end{titlepage}
\clearpage

\tableofcontents

\newpage

%-----------------------------------
\section{Objectif}
%-----------------------------------

L'objectif de ce projet est d'automatiser la balance des blancs d'un ensemble de 12 images.
La première partie consiste à détecter la présence et la position des mires MacBeth dans chaque image.
Le second objectif est de se servir de ces mires pour estimer la balance des blancs de l'image.

%-----------------------------------
\section{Données}
%-----------------------------------

Pour ce projet, un jeu de photographies a été fourni, contenant 12 images contenant des mires MacBeth. Ces images ont été prises dans des conditions d'éclairage variées, ce qui permet d'étudier l'impact de la lumière sur la perception des couleurs.

\begin{figure}[H]
    \centering
    \begin{minipage}{0.48\textwidth}
        \centering
        \includegraphics[width=\linewidth]{images/1.jpg}
        \caption{Photographie numéro 1}
    \end{minipage}
    \hfill
    \begin{minipage}{0.48\textwidth}
        \centering
        \includegraphics[width=\linewidth]{images/3.jpg}
        \caption{Photographie numéro 3}
    \end{minipage}
\end{figure}

À noter que l'image 9 comporte une mire différente des autres, ce qui empêche sa détection par la fonction OpenCV.

\begin{figure}[H]
    \centering
    \begin{minipage}{0.48\textwidth}
        \centering
        \includegraphics[width=\linewidth]{images/9.jpg}
        \caption{Photographie numéro 9}
    \end{minipage}
    \hfill
    \begin{minipage}{0.48\textwidth}
        \centering
        \includegraphics[width=\linewidth]{images/9_chart.jpg}
        \caption{Mire image numéro 9}
    \end{minipage}
\end{figure}

\clearpage

%-----------------------------------
\section{Étapes}
%-----------------------------------

La démarche suivie dans ce projet se décline en plusieurs étapes :

\subsection{Détection de la mire}

Dans un premier temps, on détecte les mires MacBeth dans chaque image, en utilisant la fonction OpenCV \href{https://docs.opencv.org/4.x/dd/d19/group__mcc.html\#gga836ee96afcefd4f35e95760ca9e8163da3c0b5a40e1157d57f944cab818e7311d}{cv2.mcc.CCheckerDetector}.

\begin{figure}[H]
    \centering
    \includegraphics[width=0.5\linewidth]{images/det_1.jpg}
    \caption{Résultat de la détection de la mire sur l'image 1}
\end{figure}

Cette fonction a toutefois quelques limites, notamment dans la détection d'une mire qui ne correspond pas tout à fait à la mire MacBeth, comme c'est le cas pour l'image 9. 
On trouve également quelques erreurs dans l'orientation de la mire, comme sur l'image 10, cette erreur en particulier ne nous permet pas d'effectuer le calcul détaillé en section~\ref{sec:balance_des_blancs}.

\begin{figure}[H]
    \centering
    \includegraphics[width=0.5\linewidth]{images/det_10.jpg}
    \caption{Mauvaise orientation de la mire sur l'image 10}
\end{figure}

\subsection{Linéarisation des images JPG}

Ensuite, il est nécessaire de linéariser les images sources. Les images JPG sont encodées avec une correction gamma, qui doit être "inversée" pour travailler avec des valeurs linéaires. Pour les images JPG suivant l'espace colorimétrique sRGB, la fonction de linéarisation s'exprime comme suit :

\begin{equation}
RGB_{\text{linéaire}} = 
\begin{cases} 
\frac{RGB_{\text{non-linéaire}}}{12.92} & \text{si } RGB_{\text{non-linéaire}} \leq 0.04045 \\ 
\left(\frac{RGB_{\text{non-linéaire}} + 0.055}{1.055}\right)^{2.4} & \text{si } RGB_{\text{non-linéaire}} > 0.04045
\end{cases}
\end{equation}

Il est bon de noter ici que la linéarisation est appliquée vectoriellement sur l'image, c'est-à-dire que le traitement s'exécute plan par plan (B, G et R) de l'image.

\begin{figure}[H]
    \centering
    \begin{minipage}{0.48\textwidth}
        \centering
        \includegraphics[width=\linewidth]{images/1.jpg}
        \caption{Photographie numéro 1 non-linéarisée}
    \end{minipage}
    \hfill
    \begin{minipage}{0.48\textwidth}
        \centering
        \includegraphics[width=\linewidth]{images/lin_1.jpg}
        \caption{Mire image numéro 1 linéarisée}
    \end{minipage}
\end{figure}

\subsection{Balance des blancs}
\label{sec:balance_des_blancs}

Le troisième et dernier objectif est de récupérer un patch gris donné de la mire et d'utiliser ses composantes BGR pour corriger l'image.
Pour ce faire, on se base sur le postulat qu'un patch de la ligne des gris (par exemple le n°21) doit avoir trois composantes égales.
Alors, une fois que nous avons récupéré ses composantes, il est possible d'effectuer les calculs suivants :

\begin{equation}
    \overline{\text{BGR}} = \frac{B + G + R}{3}
\end{equation}

Et pour chaque pixel de l'image, on applique la correction suivante :

\begin{equation}
    \text{B} = \frac{\text{B} \cdot \overline{\text{BGR}}}{B_{\text{patch}}}
\end{equation}
\begin{equation}
    \text{G} = \frac{\text{G} \cdot \overline{\text{BGR}}}{G_{\text{patch}}}
\end{equation}
\begin{equation}
    \text{R} = \frac{\text{R} \cdot \overline{\text{BGR}}}{R_{\text{patch}}}
\end{equation}

Il est important de noter que la moyenne du patch gris est calculée sur un voisinage de 3x3 pixels, afin d'éviter les effets de bord.

\clearpage

%-----------------------------------
\section{Résultats}
%-----------------------------------

On arrive aux images suivantes :

\begin{figure}[H]
    \centering
    \begin{minipage}{0.48\textwidth}
        \centering
        \includegraphics[width=\linewidth]{images/data/1.jpg}
        \caption{Image 1 originale}
    \end{minipage}
    \hfill
    \begin{minipage}{0.48\textwidth}
        \centering
        \includegraphics[width=\linewidth]{images/wb/balanced_1.jpg}
        \caption{Image 1 après balance des blancs}
    \end{minipage}
\end{figure}

\begin{figure}[H]
    \centering
    \begin{minipage}{0.48\textwidth}
        \centering
        \includegraphics[width=\linewidth]{images/data/2.jpg}
        \caption{Image 2 originale}
    \end{minipage}
    \hfill
    \begin{minipage}{0.48\textwidth}
        \centering
        \includegraphics[width=\linewidth]{images/wb/balanced_2.jpg}
        \caption{Image 2 après balance des blancs}
    \end{minipage}
\end{figure}

\begin{figure}[H]
    \centering
    \begin{minipage}{0.48\textwidth}
        \centering
        \includegraphics[width=\linewidth]{images/data/3.jpg}
        \caption{Image 3 originale}
    \end{minipage}
    \hfill
    \begin{minipage}{0.48\textwidth}
        \centering
        \includegraphics[width=\linewidth]{images/wb/balanced_3.jpg}
        \caption{Image 3 après balance des blancs}
    \end{minipage}
\end{figure}

\begin{figure}[H]
    \centering
    \begin{minipage}{0.48\textwidth}
        \centering
        \includegraphics[width=\linewidth]{images/data/4.jpg}
        \caption{Image 4 originale}
    \end{minipage}
    \hfill
    \begin{minipage}{0.48\textwidth}
        \centering
        \includegraphics[width=\linewidth]{images/wb/balanced_4.jpg}
        \caption{Image 4 après balance des blancs}
    \end{minipage}
\end{figure}

\begin{figure}[H]
    \centering
    \begin{minipage}{0.48\textwidth}
        \centering
        \includegraphics[width=\linewidth]{images/data/5.jpg}
        \caption{Image 5 originale}
    \end{minipage}
    \hfill
    \begin{minipage}{0.48\textwidth}
        \centering
        \includegraphics[width=\linewidth]{images/wb/balanced_5.jpg}
        \caption{Image 5 après balance des blancs}
    \end{minipage}
\end{figure}

\begin{figure}[H]
    \centering
    \begin{minipage}{0.48\textwidth}
        \centering
        \includegraphics[width=\linewidth]{images/data/6.jpg}
        \caption{Image 6 originale}
    \end{minipage}
    \hfill
    \begin{minipage}{0.48\textwidth}
        \centering
        \includegraphics[width=\linewidth]{images/wb/balanced_6.jpg}
        \caption{Image 6 après balance des blancs}
    \end{minipage}
\end{figure}

\begin{figure}[H]
    \centering
    \begin{minipage}{0.48\textwidth}
        \centering
        \includegraphics[width=\linewidth]{images/data/7.jpg}
        \caption{Image 7 originale}
    \end{minipage}
    \hfill
    \begin{minipage}{0.48\textwidth}
        \centering
        \includegraphics[width=\linewidth]{images/wb/balanced_7.jpg}
        \caption{Image 7 après balance des blancs}
    \end{minipage}
\end{figure}

\begin{figure}[H]
    \centering
    \begin{minipage}{0.48\textwidth}
        \centering
        \includegraphics[width=\linewidth]{images/data/8.jpg}
        \caption{Image 8 originale}
    \end{minipage}
    \hfill
    \begin{minipage}{0.48\textwidth}
        \centering
        \includegraphics[width=\linewidth]{images/wb/balanced_8.jpg}
        \caption{Image 8 après balance des blancs}
    \end{minipage}
\end{figure}

\begin{figure}[H]
    \centering
    \begin{minipage}{0.48\textwidth}
        \centering
        \includegraphics[width=\linewidth]{images/data/10.jpg}
        \caption{Image 10 originale}
    \end{minipage}
    \hfill
    \begin{minipage}{0.48\textwidth}
        \centering
        \includegraphics[width=\linewidth]{images/wb/balanced_10.jpg}
        \caption{Image 10 après balance des blancs}
    \end{minipage}
\end{figure}

\begin{figure}[H]
    \centering
    \begin{minipage}{0.48\textwidth}
        \centering
        \includegraphics[width=\linewidth]{images/data/11.jpg}
        \caption{Image 11 originale}
    \end{minipage}
    \hfill
    \begin{minipage}{0.48\textwidth}
        \centering
        \includegraphics[width=\linewidth]{images/wb/balanced_11.jpg}
        \caption{Image 11 après balance des blancs}
    \end{minipage}
\end{figure}

\begin{figure}[H]
    \centering
    \begin{minipage}{0.48\textwidth}
        \centering
        \includegraphics[width=\linewidth]{images/data/12.jpg}
        \caption{Image 12 originale}
    \end{minipage}
    \hfill
    \begin{minipage}{0.48\textwidth}
        \centering
        \includegraphics[width=\linewidth]{images/wb/balanced_12.jpg}
        \caption{Image 12 après balance des blancs}
    \end{minipage}
\end{figure}

\clearpage

%-----------------------------------
\section{Où trouver notre travail ?}
%-----------------------------------

Tout le travail dont il est question dans ce rapport est disponible sur \href{https://github.com/antoinedenovembre/colorimetry_project_2}{github}.

\end{document}
